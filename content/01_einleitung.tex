\chapter{Einleitung}
Siliziumdetektoren sind in der Wissenschaft und in der Industrie weit verbreitete
Detektoren. In der Hochenergiephysik werden sie in Collider-Experimenten, wie ATLAS, verwendet.
Sie  lassen präzise Rückschlüsse auf die Energie, sowie den Impuls von Teilchen durch eine
hohe Ortsauflösung zu.


ATLAS verwendet Siliziumdetektoren als
einen Teil seines inneren Detektors zur Bestimmung von Richtung, Impuls
und Ladung von geladenen Teilchen. Es ist eines der vier großen Experimente am LHC.
Der innere Detektor ist zylinderförmig mit einer Länge von $\SI{6.2}{\meter}$ und
einem Radius von $\SI{2.1}{\meter}$. Er wird von einem $\SI{2}{\tesla}$ Magnetfeld, eines ihn umgebenden Solenoids,
durchsetzt und krümmt Teilchenbahnen in Abhängigkeit von Ladung und Impuls.
Bestehend aus drei Komponenten, dem Pixeldetektor, dem "Semiconductor Tracker" (SCT) und dem "Transition Radiation Tracker" (TRT) kommen  die
Siliziumdetektoren in den ersten beiden Bestandteilen zum Einsatz. Im SCT werden, anders als im Pixeldetektor, Siliziumstreifen-Sensoren verwendet.
Mit einer Fläche von $\SI{1.7}{\meter\squared}$, drei Lagen und insgesamt 1744 Modulen ist es im Pixeldetektor möglich
Information über die Strecke der entstehenden Teilchen zu erlangen. Der SCT besteht aus 4088 Modulen über vier zylindrische Lagen,
sowie 18 scheibenförmige Lagen, verteilt und umfasst eine Fläche von $\SI{63}{\meter\squared}$. Auch dieser wird für
die Rekonstruktion der Bahn der Teilchen verwendet, jedoch auch für die Identifikation und der Bestimmung des Impulses der Teilchen.
Der TRT wird ebenfalls zur Identifikation von Teilchen, hauptsächlich Elektronen und Pionen, verwendet und besteht aus 73 Lagen
von Straw-Detektoren (viele aneinander gereihte Proportionalzählrohre).
Weiterreichende Informationen über ATLAS befinden sich in \cite{ATLAS}.
Für Collider-Experimente sind möglichst funktionsfähige Detektoren essenziell. Da diese während des Experimentes und starker Bestrahlung
stehen ist eine Untersuchung der Folgen und Gegenmaßnahmen notwendig.


Einfallende Teilchen können die
Struktur des Detektors und somit dessen Dotierungskonzentration verändern, wodurch dieser ineffizienter wird.
Es ist somit notwendig, die Strahlenschäden möglichst gering zu halten damit
die Halbleiter eine möglichst lange Lebenszeit besitzen.
Die Bestrahlung der Detektoren für einen anwachsenden Leckstrom, welcher den
gemessenen elektrischen Signalen überlagert ist und ein Rauschen kreiert. Ein
effektiver Detektor sollte somit einen möglichst geringen Leckstrom haben.
Für einen Detektor ist eine möglichst große Depletionszone ebenfalls wichtig, idealerweise
liegt eine vollständige Depletion des Detektorvolumens vor. Um dies zu erreichen wird
eine externe Spannung verwendet. Mit steigender Spannung wächst jedoch auch der Leckstrom, weshalb
darauf geachtet werden muss bei möglichst geringer externer Spannung den Detektor zu depletieren.
Dies ist abhängig von der Dotierungskonzentration, da sie die Ladung der
Raumladungszone beschreibt.

Die durch Strahlung induzierte Änderung der Dotierungskonzentration und des Leckstromes sind
somit wichtige Größen für die Qualität eines Detektors.
Um die Schäden zu verringern wird der Annealingeffekt ausgenutzt. Mit diesem lassen
sich Defekte durch Wärme ausheilen, aber auch verschlimmern.
Manche Schäden sind durch Annealing nicht
aufhebbar, wodurch die Detektoren unweigerlich mit fortlaufendem Einsatz
an Leistungsfähigkeit verlieren.

In dieser Arbeit wird das
theoretische Verhalten der beiden Größen auf der Grundlage des Hamburger-Modells betrachtet.
Ein Programm zur Modellierung von Annealingeffekten kann somit zur
Optimierung des Annealingprozesses beitragen.
