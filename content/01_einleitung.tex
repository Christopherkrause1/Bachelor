\chapter{Einleitung}
Siliziumdetektoren sind in der Hochenergiephysik und in der Industrie weit verbreitete
Detektoren, welche in Collider-Experimenten, wie Atlas, verwendet werden.
Sie operieren bei Raumtemperatur und können präzise Rückschlüsse auf die
Energie von Teilchen schließen. Einfallende Teilchen können die
Struktur des Detektors verändern, wodurch dieser ineffizienter wird.
Es ist somit notwendig, die Strahlenschäden möglichst gering zu halten damit
die Halbleiter eine möglichst lange Lebenszeit besitzen. Manche Schäden sind durch annealing nicht
aufhebbar, wodurch die Detektoren unweigerlich mit fortlaufendem Einsatz
an Leistungsfähigkeit verlieren. Um die reparablen Schäden im Siliziumkristall
zu beseitigen wird der Annealingeffekt genutzt. Hierbei wird durch zugeführte Wärme gewisse
Defekte entfernt.
Zusätzlich sorgt die Bestrahlung der Detekoren für einen anwachsenden Leckstrom, welcher den
gemessenen elektrischen Signalen überlagert ist und ein Rauschen kreiert. Ein
effektiver Detektor sollte somit einen möglichst geringen Leckstrom haben.
Für einen Detektor ist eine möglichst große Depletionszone ebenfalls wichtig, idealerweise
liegt eine vollständige Depletion des Detekorvolumens vor. Um dies zu erreichen wird
eine externe Spannung verwendet. Mit steigender Spannung wächst jedoch auch der Leckstrom, weshalb
darauf geachtet werden muss bei möglichst geringer externer Spannung den Detektor zu depletieren.
Dies ist abhängig von der Dotierungskonzentration, da sie die Ladung der
Raumladungszone beschreibt.

Die durch Strahlung induzierte Änderung der Dotierungskonzentration und des Leckstromes sind
somit wichtige Größen für die Qualität eines Detektors. In dieser Arbeit wird das
theoretische Verhalten der beiden Größen auf der Grundlage des Hamburger-Modells betrachtet.
Ein Programm zur Modellierung von Annealingeffekten kann somit zur
optimierung des Annealingprozesses beitragen.


Das oben erwähnte Collider-Experiment, ATLAS, verwendet Siliziumdetekoren als
einen Teil seines inneren Detektors zur Bestimmung von Richtung, Impuls
und Ladung von geladenen Teilchen. Es ist Teil des LHC am Cern und umfasst
das größte Volumen aller Detektoren, die bisher gebaut wurden\cite{atlas}. 
