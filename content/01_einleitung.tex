\chapter{Einleitung}
Siliziumdetektoren sind in der Wissenschaft und in der Industrie weit verbreitete
Detektoren. In der Hochenergiephysik werden sie in Collider-Experimenten, wie ATLAS, verwendet.
Sie  lassen präzise Rückschlüsse auf die Ladung sowie den Impuls von Teilchen durch eine
hohe Ortsauflösung zu.


Im ATLAS Experiment werden Siliziumdetektoren, zur Bestimmung von Richtung, Impuls
und Ladung von geladenen Teilchen, verwendet. Es ist eines der vier großen Experimente am LHC.

Der innere Detektor ist der erste Bestandteil des ATLAS Experimentes zur Analyse der Zerfallsprodukte der
kollidierenden Teilchen. Er
wird von einem $\SI{2}{\tesla}$ Magnetfeld eines ihn umgebenden Solenoids
durchsetzt und krümmt Teilchenbahnen in Abhängigkeit von Ladung und Impuls.

Der "Transition Radiation Tracker" (TRT), der "Semiconductor Tracker" (SCT) und
der Pixeldetektor, sind die drei Komponenten
des inneren Detektors. Siliziumdetektoren kommen lediglich in den letzten beiden Bestandteilen vor.

Der TRT wird zur Identifikation von Teilchen, hauptsächlich Elektronen und Pionen, verwendet und besteht aus 73 Lagen
von Straw-Detektoren (viele aneinander gereihte Proportionalzählrohre).

Im SCT werden Siliziumstreifendetektoren verwendet und er dient zur
Rekonstruktion der Bahn der Teilchen, jedoch auch für die Identifikation und der Bestimmung des Impulses der Teilchen.

Mit einer Fläche von $\SI{1.7}{\meter\squared}$ und drei Lagen von Modulen
%und insgesamt 1744 Modulen
ist es im Pixeldetektor möglich
Informationen über die Strecke der entstehenden Teilchen zu erlangen.
Weiterreichende Informationen über das ATLAS-Experiment befinden sich in \cite{ATLAS}.
Für Collider-Experimente sind möglichst funktionsfähige Detektoren essenziell.
Die Detektoren der innersten Schicht des Pixeldetektors (IBL)
werden bei den Experimenten mit Fluenzen von ungefähr $\Phi_{\mathrm{eq}} =3,3\cdot 10^{15} \, \mathrm{n_{eq}/cm^2}$ bestrahlt \cite{Capeans:1291633}.
Die erwartete Fluenz des geplanten "Inner Tracker" (ITK) beträgt
$\Phi_{\mathrm{eq}} =1,3\cdot 10^{16} \, \mathrm{n_{eq}/cm^2}$ \cite{itk}. Eine Untersuchung
der Folgen der Strahlenbeschädigung ist somit notwendig.

%Da diese während des Experimentes unter starker Bestrahlung
%stehen ist eine Untersuchung der Folgen und Gegenmaßnahmen notwendig. Die Sensoren der innersten Schicht des Pixeldetektors (IBL)
%werden bei Experimenten mit Fluenzen von ungefähr $\Phi_{\mathrm{eq}} =5\cdot 10^{15} \, \mathrm{n_{eq}/cm^2}$ bestrahlt \cite{Capeans:1291633}.
%Die erwartete Fluenz des geplanten \textit{Inner tracker} (ITK) des Pixeldetektors beträgt
%$\Phi_{\mathrm{eq}} =1,2\cdot 10^{15} \, \mathrm{n_{eq}/cm^2}$ \cite{itk}.
Es handelt sich bei den Sensoren um Dioden.
Einfallende Teilchen können die
Struktur dieser verändern, wodurch sie ineffizienter werden.
Es ist somit notwendig die Strahlenschäden möglichst gering zu halten, damit
die Sensoren eine möglichst lange Lebenszeit besitzen.


Die Bestrahlung der Sensoren hat zwei Effekte zur Folge. Zum einen wächst der Leckstrom, welcher den
gemessenen elektrischen Signalen überlagert ist und ein Rauschen kreiert. Ein
effektiver Sensor sollte somit einen möglichst geringen Leckstrom haben.

Für einen Sensor ist eine möglichst große Depletionszone ebenfalls wichtig,
da Teilchen nur beim Durchdringen der Depletionszone über ein elektrisches Signal detektiert
werden können.
Idealerweise
liegt eine vollständige Depletion des Sensorvolumens vor. Um dies zu erreichen wird
eine externe Spannung verwendet. Mit steigender Spannung wächst jedoch auch der Leckstrom, weshalb
darauf geachtet werden muss bei möglichst geringer externer Spannung den Sensor zu depletieren.
Dies ist abhängig von der Dotierungskonzentration, da sie die Ladung der
Raumladungszone beschreibt.

Die durch Strahlung induzierte Änderung der effektiven Dotierungskonzentration und des Leckstromes sind
somit wichtige Größen für die Qualität eines Sensors.
Um die Schäden zu verringern wird der Annealingeffekt ausgenutzt. Mit diesem lassen sich
Defekte durch Wärme verändern und teilweise beheben. Die Eigenschaften des Sensors können
dabei sowohl verbessert als auch bei zu langer Erwärmung verschlechtert werden.
Jedoch kann gezieltes Annealing die durch Bestrahlung induzierten Schäden im Sensor nur
abschwächen und nicht vollständig beheben.
Um zu überprüfen, ob Sensoren den Anforderungen im Experiment genügen, werden sie in
Bestrahlungseinrichtungen gezielt bestrahlt.

In dieser Arbeit wird das
theoretische Verhalten der Dotierungskonzentration und des Leckstroms auf der Grundlage des Hamburger Modells betrachtet.
Ein Programm zur Modellierung von Annealingeffekten kann somit zur
Optimierung durch die Bewertung von Annealingschritten und der Vorhersage der Effekte beitragen.

%Zusätzlich ist es an keine weiteren Programme gebunden und unkompliziert bedienbar.
