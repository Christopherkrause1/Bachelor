\chapter{Einleitung}
Siliziumdetektoren sind in der Hochenergiephysik und in der Industrie weit verbreitete
Detektoren, welche in Collider-Experimenten, wie Atlas, verwendet werden.
Sie operieren bei Raumtemperatur und können präzise Rückschlüsse auf die
Energie von Teilchen schließen. Einfallende Teilchen können die
Struktur des Detektors verändern, wodurch dieser ineffizienter wird.
Es ist somit notwendig, die Strahlenschäden möglichst gering zu halten damit
die Halbleiter eine möglichst lange Lebenszeit besitzen. Manche Schäden sind
irreparabel, wodurch die Detektoren unweigerlich mit fortlaufendem Einsatz
an Leistungsfähigkeit verlieren. Um die reparablen Schäden im Siliziumkristall
zu beseitigen wird der Annealingeffekt genutzt. Hierbei wird durch zugeführte Wärme gewisse
Defekte entfernt.

Das oben erwähnte Collider-Experiment, ATLAS, verwendet Siliziumdetekoren als
einen Teil seines inneren Detektors zur Bestimmung von Richtung, Impuls
und Ladung von geladenen Teilchen. Es ist Teil des LHC am Cern und umfasst
das größte Volumen aller Detektoren, die bisher gebaut wurden\cite{atlas}.
