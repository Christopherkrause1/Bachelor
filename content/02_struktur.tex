\chapter{Siliziumdetektoren}
Die Siliziumhalbleiter sind in der hochenenergie Physik weit verbreitete
Detektoren, welche in Collider-Experimenten , wie Atlas, verwendet werden.
Sie operieren bei Raumtemperatur und können präzise Rückschlüsse auf die
Energie von Teilchen schließen.

\section{Halbleiter}
Die im Halbleiter vorkommenden Elektronen des Siliziumkristalls wechselwirken mit anderen Elektronen und
Kernen, wodurch es zu vielen Aufspaltungen des Energieniveaus dieser kommt. Die
Zustände liegen dicht bei einander, weshalb von Energiebändern gesprochen wird.
Das höchste von den Elektronen besetzte Band ist das Valenzband, welches von dem
nächsthöheren Band durch eine Bandlücke getrennt ist. Elektronen können
durch Anregungen in das Leitungsband wandern und somit Strom leiten, vorrausgesetzt
die Anregungen sind größer als die Energielücke.

Das Einbringen von Fremdatomen in den Siliziumkristall kann dessen Eigenschaften ändern
und wird Dotierung genannt.
Hat das Fremdatom mehr oder weniger Valenzelektronen als das Silizium, kommt
es zu weiteren Aufspaltungen der Energienivaus, welche sich in der Energielücke
befindet. Dadurch wird es für Elektronen einfacher in das Leitungsband zu wandern.
Bei den Fremdatomen wird zwischen Donatoren und Akzeptoren unterschieden, wobei die
Ersteren mehr Valenzelektronen als das Silizium besitzen und die Zweiteren weniger.
Bereiche des Siliziumkristalls mit Donatoren werden n-Dotierte Halbleiter
und Bereiche mit Akzeptoren p-Dotierte Halbleiter genannt.

Bei einem Übergang von einer n-Dotierten zu einer p-Dotierten Schicht rekombinieren
die Elektronen der n-Dotierten Schicht mit den Elektronenlöchern der p-dotierten Schicht, da
sie in die anderen Bereiche diffundieren. Durch die übirg bleibenden Ortsfesten Atomkerne
baut sich ein elektrisches Feld auf, welches dem Diffusionsstrom entgegen wirkt.
Ein sich einstellendes Gleichgewicht führt zu einer raumladungsfreien Zone bei
dem p-n-Übergang.
