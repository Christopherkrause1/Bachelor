\chapter{Theoretische Grundlagen}


\section{Siliziumhalbleiterdetektoren}

Die Siliziumhalbleiter sind in der hochenenergie Physik weit verbreitete
Detektoren, welche in Collider-Experimenten , wie Atlas, verwendet werden.
Sie operieren bei Raumtemperatur und können präzise Rückschlüsse auf die
Energie von Teilchen schließen.


Die im Halbleiter vorkommenden Elektronen des Siliziumkristalls wechselwirken mit anderen Elektronen und
Kernen, wodurch es zu vielen Aufspaltungen des Energieniveaus dieser kommt. Die
Zustände liegen dicht bei einander, weshalb von Energiebändern gesprochen wird.
Das höchste von den Elektronen besetzte Band ist das Valenzband, welches von dem
nächsthöheren Band durch eine Bandlücke getrennt ist. Elektronen können
durch Anregungen in das Leitungsband wandern und somit Strom leiten, vorrausgesetzt
die Anregungen sind größer als die Energielücke.

Das Einbringen von Fremdatomen in den Siliziumkristall kann dessen Eigenschaften ändern
und wird Dotierung genannt.
Hat das Fremdatom mehr oder weniger Valenzelektronen als das Silizium, kommt
es zu weiteren Aufspaltungen der Energieniveaus, welche sich in der Energielücke
befindet. Dadurch wird es für Elektronen einfacher in das Leitungsband zu wandern.
Bei den Fremdatomen wird zwischen Donatoren und Akzeptoren unterschieden, wobei die
Ersteren mehr Valenzelektronen als das Silizium besitzen und die Zweiteren weniger.
Bereiche des Siliziumkristalls mit Donatoren werden n-Dotierte Halbleiter
und Bereiche mit Akzeptoren p-Dotierte Halbleiter genannt.

Bei einem Übergang von einer n-Dotierten zu einer p-Dotierten Schicht rekombinieren
die Elektronen der n-Dotierten Schicht mit den Elektronenlöchern der p-dotierten Schicht, da beide
in die anderen Bereiche diffundieren. Durch die übrig bleibenden Ortsfesten Atomkerne
baut sich ein elektrisches Feld auf, welches dem Diffusionsstrom entgegen wirkt.
Ein sich einstellendes Gleichgewicht führt zu einer raumladungsfreien Zone bei
dem p-n-Übergang.
Durch eine äußere Spannung wird die Zone vergrößert und dient als Ionisationskammer für einfallende Teilchen.
Die entstehenden Elektronen-Loch Paare werden als elektrisches Signal gemessen.


\section{Strahlenschäden}
Wechselwirkungen von hochenergetischen Teilchen mit Siliziumhalbleitern können zu Defekten in deren
Gitterstruktur führen.
Die Energie der Teilchen an die Gitteratome wird dabei durch Ionisation und elastischer Streuung abgegen.
Es muss zwischen verschiedenen Arten von Defekten unterschieden werden. Die Teilchen können ein Gitteratom aus dem
Gitter herausschlagen, wodurch eine Leerstelle und ein Zwischengitteratom entsteht. Das herausgeschlagene Atom
kann Silizium, aber auch ein Fremdatom sein. Daraus folgt eine Änderung der Dotierungskonzentration im
Halbleiter und somit eine Änderung seiner Eigenschaften.

Um Fluenzen $ \Phi$ von verschiedenen Teilchen (Lepton, Hadron, Photon) miteinander vergleichen zu können, werden diese
auf eine $1$MeV Neutronstrahlung normalisiert und haben die Einheit $\mathrm{\frac{n_{\mathrm{eq}}}{cm^2}}$.


\section{Annealing}
Annealing beschreibt allgemein die durch Erhitzung hervorgerufenen Veränderungen von Materialeigenschaften. Für die
Siliziumhalbleiter sollen dabei, durch die zugeführte Wärme, die Defekte teilweise behoben werden um somit eine
bessere Funktionsfähigkeit über einen längeren Zeitraum zu garantieren. Die auftetenden Effekte sind dabei wie folgt:

\textbf{1. Migration:} Ist die Temperatur hoch genug können die Zwischengitteratome durch das Gitterwandern und
zum einen Leerstellen im Gitter füllen.

\textbf{2. Komplexformation:} Das migrieren der Atome kann zum Ausbilden von Komplexen führen. So kann sich zum Beispiel ein
Zwischengittersiliziumatom mit einem Siliziumatom im Gitter binden.

\textbf{3. Dissoziation:} Komplexe können bei hohen Temperaturen Dissoziieren, wodurch einzelne Bestandteile des Komplexes
durch das Gitter migrieren können. Dafür muss die Schwingungsenergie des Gitters größer als die Bindungsenergie der Komplexe sein.

Die Annealingeffekte bewirken eine Änderung in der Dotierungskonzentration $N_{\mathrm{eff}}= |N_{\mathrm{D}}-N_{\mathrm{A}}|$, welches durch das Hamburger Modell beschrieben wird.
Die Änderung von $N_{\mathrm{eff}}$ ist gegeben durch:
\begin{align}
  \Delta N_{\mathrm{eff}}(t, \Phi_{\mathrm{eq}}, T)   = N_{\mathrm{C}}(\Phi_{\mathrm{eq}}) + N_{\mathrm{A}}(t, \Phi_{\mathrm{eq}}, T) + N_{\mathrm{Y}}(t, \Phi_{\mathrm{eq}}, T)
\end{align}
Die einzelnen Summanden werden im folgenden erläutert.

\textbf{\textit{Stable Damage}} $\symbf{N_{\mathrm{C}}}:$ Der stable Damage ist unabhängig von der Temperatur und wird durch das annealing nicht beeinflusst.
\begin{align}
  N_{\mathrm{C}}(\Phi_{\mathrm{eq}}) = N_{\mathrm{C0}}[1-\exp{(-c \cdot \Phi_{\mathrm{eq}})}] + g_{\mathrm{c}} \cdot \Phi_{\mathrm{eq}}
\end{align}
Der erste Summand folgt aus einer unvollständigen Entfernung von Donatoren, der zweite Summand beschreibt stabile Defekte, welche sich wie Akzeptorenverhalten \cite{beyer}.


\textbf{\textit{Shortterm annealing}} $\symbf{N_{\mathrm{A}}}:$ Das shortterm annealing bewirkt eine Verringerung der Dotierungskonzentration und entsteht durch
das Annealing von Akzeptoren, was zu einer kleineren Dotierungskonzentration führt. . Dieser Effekt soll
zur Ausheilung des Sensors verwendet werden. Die Bezeichnung rührt daher, dass $N_{\mathrm{A}}$ meistens in den ersten 100 Minuten verschwindend gering wird. Die
Funktion wird näherungsweise beschrieben durch:
\begin{align}
  N_{\mathrm{A}}(t, \Phi_{\mathrm{eq}}, T) = \Phi_{\mathrm{eq}} \cdot g_{\mathrm{a}} \cdot \exp{\left(-\frac{t}{\tau_{\mathrm{a}}}\right)}
\end{align}


\textbf{\textit{Longterm annealing}} $\symbf{N_{\mathrm{A}}}:$
