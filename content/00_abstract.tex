\thispagestyle{plain}

\section*{Kurzfassung}
In dieser Arbeit wird die Entwicklung eines Programmes zur
Berechnung von Annealingeffekten der Dotierungskonzentration und der Schadensrate in bestrahlten
Silizumdetektoren
beschrieben. Auf der Grundlage des Hamburg Modells modelliert das Programm die Annealingeffekte für beliebige
Temperaturverläufe und konstante Fluenzen.
Dabei können Parameter des Modells frei eingestellt werden.
Die hierdurch berechneten Schadensraten basieren auf bekannten Temperaturverläufen und werden mit den gemessenen Daten verglichen,
um die Validität des Programmes zu überprüfen. Sichtbar werdende Abweichungen der
experimentellen Daten von dem theoretischen Modell lassen sich durch experimentelle Unsicherheiten erklären.
Zusätzlich wird ein Interface vorgestellt, welches Annealingeffekte für konstante Temperaturen und Fluenzen
über beliebige Zeiträume modelliert.

\section*{Abstract}
\begin{english}
  This thesis describes the development of a program for the calculation of annealing effects of the doping concentration and the damage rate.
  Based on the Hamburg model, the program models the annealing effects for silicon detectors for any temperature profile and constant fluences.
  Parameters of the used function can be set freely.
  The theoretically calculated damage rates are based on known temperature profiles and are compared with exmerimental data,
  to check the validity of the program.
   Visible deviations of the experimental data from the theoretical model can be explained by experimental uncertainties. In addition, an interface
   is presented which provides annealing effects for constant temperatures and fluences over any time period.



\end{english}
