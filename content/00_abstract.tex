\thispagestyle{plain}

\section*{Kurzfassung}
In dieser Arbeit wird die Entwicklung eines Programmes zur
Berechnung von Annealingeffekten der Dotierungskonzentration und der Schadensrate
beschrieben. Auf der Grundlage des Hamburg-Modells modelliert das Programm für beliebige
Temperaturverläufe und konstante Fluenzen die Annealingeffekte für Siliziumdetektoren.
Dabei können Parameter der verwendeten Funktion frei eingestellt werden.
Die hierdurch theoretisch berechneten Schadensraten werden mit den experimentellen Daten verglichen
um die Validität des Programmes zu überprüfen. Sichtbar werdende Abweichungen der
experimentellen Daten von dem theoretischen Modell lassen sich durch experimentelle Unsicherheiten erklären.
Zusätzlich wird ein Interface vorgestellt, welches Annealingeffekte für konstante Temperaturen und Fluenzen
über beliebige Zeiträume modelliert.

\section*{Abstract}
\begin{english}
  This bachelor thesis describes the procedure of a program written in Python for the
  calculation of annealing effects of the doping concentration and the damage rate.
  On the basis of the Hamburg-model, the program is to be used for any
  temperature curves and constant fluences that approximate annealing effects for silicon detectors.
  Parameters of the used function and linear interpolation intervals for
  additional data can be freely set.
  The damage rates calculated theoretically in this way are compared with the experimental data
  to check the validity of the program. Deviations of the
  theoretical and experimental values can be determined by uncertainties in the leakage current,
  the volume, as well as by previous annealing of the detectors.
  In addition, an interface is presented that allows the calculation of annealing effects for constant temperatures and fluences
  over any time period.

\end{english}
