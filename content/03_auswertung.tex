\chapter{Wichtige Hinweise zum Dokument}\label{make}
\section{Annealingeffekt bei konstanten Temperaturen}
Gleichung \ref{eqn:N_eff} wird für das Modellieren der Annealingeffekte verwendet. In
Abbildung \ref{fig:N_eff} und \ref{fig:damage} sind diese, jeweils für zwei Temperaturen dargestellt.
Für alle Plots werden materialabhängige
Parameter einer "WE-25k" aus \cite{moll} genommen.

\begin{figure}
    \includegraphics[width=0.95\textwidth]{build/annealing.PDF}
    \caption{$\Delta N_{eff}$ einer WE-25k Diode nach Bestrahlung mit Fluenz $\Phi_{\mathrm{eq}} = \SI{5e15}{\per\centi\meter\squared}.$}
    \label{fig:N_eff}
\end{figure}

\begin{figure}
    \includegraphics[width=0.95\textwidth]{build/damage.PDF}
    \caption{Schadensrate einer WE-25k Diode.}
    \label{fig:damage}
\end{figure}
