\chapter{Entwicklung des Programmes zur Berechnung von Annealingeffekten}\label{make}
Die folgenden Unterkapitel beschreiben die Vorgehensweise für die Entwicklung  eines Programmes zur
Berechnung von Annealingeffekten. Das Programm ist in der Programmiersprache Python unter
der Verwendung der \textit{NumPy} \cite{oliphant2006guide}, \textit{Matplotlib} \cite{Hunter:2007} und \textit{math} Bibliotheken geschrieben.
\section{Annealingeffekte für konstante Temperaturen}
Zur Überprüfung der Implementation der Funktionen werden
Gleichung \ref{eqn:N_eff} und \ref{eqn:damage}  bei konstanter Temperatur verwendet. In
Abbildung \ref{fig:N_eff} und \ref{fig:damage} sind diese, für $10^5$ Minuten bei einer Temperatur
von $\SI{60}{\celsius}$ und $\SI{80}{\celsius}$  dargestellt.
Für alle Plots werden Fitparameter
einer "WE-25k" Diode aus \cite{moll} genommen und sind in Tabelle \ref{tab:w1} dargestellt. Die verwendeten
Parameter sind beispielhaft gewählt und können gezielt angepasst werden.

%\begin{table}[H]
%\centering
%\caption{Positionen der Photodiode und gemessene Stromstärken}
%\sisetup{table-format=2.1}
%\begin{tabular}{S S S S S S}
%  \toprule
%    \multicolumn{5}{c}{Fitparameter} & \multicolumn{1}{c}{Materialparameter} \\
%    \cmidrule(lr){1-5}\cmidrule(lr){6-6}
%    {$\alpha_{\mathrm{I}}/10^{-17}\, \mathrm{Acm^{-1}}$} & {$\beta/\, /10^{-17}\, \mathrm{Acm^{-18}}$} & {$k_{0\mathrm{I}}/\,10^{13}\mathrm{s^{-1}}$} &
%    {$k_{0\mathrm{a}}/\,10^{13}\mathrm{s^{-1}}$} & {$k_{Y\mathrm{I}}/\,10^{15}\mathrm{s^{-1}}$} & {$E_{\mathrm{I}}/\, \mathrm{eV}$} \\
%    \midrule
%    1 & 1 & 1 & 1 & 1 \\
%    \bottomrule
%  \end{tabular}
%\end{table}

\begin{table}
  \centering
  \caption{Fitparameter der "WE-25k" Diode für die Modellierung von Annealingeffekten }
  \label{tab:w1}
  \begin{tabular}{c c}
    \toprule
    Fitparameter & Wert  \\
    \midrule
        $\alpha_{\mathrm{I}}$  &    1,23 $\cdot$ $  10^{-17}        \mathrm{Acm^{-1}}$    \\
        $\beta $               &    3,07 $\cdot$ $10^{-17}          \mathrm{Acm^{-18}}$     \\
        $N_{\mathrm{C0}}$             &    1,1  $\cdot$            $10^{11}\mathrm{cm^{-3}}$   \\
        $c$                          &    75   $\cdot$             $10^{-14}\mathrm{cm^{-2}}$    \\
        $k_{0\mathrm{I}}$            &    1,2  $\cdot$             $10^{13}\mathrm{s^{-1}}$\\
        $k_{0\mathrm{a}}$           &    2,4  $\cdot$              $10^{13}\mathrm{s^{-1}}$   \\
        $k_{0\mathrm{Y}}$            &    1,5  $\cdot$             $10^{15}\mathrm{s^{-1}}$     \\
        $E_{\mathrm{I}}$                       &    1,11           $\mathrm{eV}$       \\
        $E_{\mathrm{aa}}$                     &    1,09            $\mathrm{eV}$       \\
        $E_{\mathrm{Y}}$                      &    1,33            $\mathrm{eV}$    \\
        $g_{\mathrm{c}}$      &    1,58 $\cdot$                    $10^{-2}\mathrm{cm^{-2}}$           \\
        $g_{\mathrm{a}}$           &    1,59 $\cdot$               $10^{-2}\mathrm{cm^{-2}}$         \\
        $g_{\mathrm{Y}}$          &    4,84 $\cdot$                $10^{-2}\mathrm{cm^{-2}}$      \\
    \bottomrule
  \end{tabular}
\end{table}
Die \textit{introduction rates} und die Fitparameter $N_{\mathrm{C0}}$, $c$, $\alpha_{\mathrm{I}}$ und $\beta $ sind dabei materialabhängig.

\begin{figure}
  \centering
    \includegraphics[width=0.82\textwidth]{build/annealing.PDF}
    \caption{$\Delta N_{\mathrm{eff}}$ einer WE-25k Diode nach einer Bestrahlung mit einer Fluenz von
    $5\cdot 10^{15} \, \mathrm{n_{\mathrm{eq}}/cm^2}$.}
    \label{fig:N_eff}
\end{figure}

\begin{figure}
  \centering
    \includegraphics[width=0.82\textwidth]{build/damage.PDF}
    \caption{Schadensrate einer WE-25k Diode.}
    \label{fig:damage}
\end{figure}

Beide Annealingeffekte stimmen mit dem Verlauf von Abbildung \ref{fig:n_eff_beispiel} und \ref{fig:damage_rates} über ein, was
die korrekte Implementierung der Funktionen \ref{eqn:N_eff} und \ref{eqn:damage}  des Programms
bestätigt.



\section{Annealingeffekte für nicht konstante Temperaturen}{\label{nicht_konstant}}
Die Gleichungen \ref{eqn:N_eff} und \ref{eqn:damage} gehen bei der Berechnung der Annealingeffekte von
einer konstanten Temperatur aus, wodurch diese nicht geeignet sind um $\Delta N_{\mathrm{eff}}$ und $\alpha$ für
Temperaturverläufe zu berechnen, welche deutlich von einer konstanten Temperatur abweichen.


%Werden die selben Gleichungen zur Modellierung von Annealingeffekte für nicht
%konstante Temperaturen verwendet, so kommt es zu deutlichen Abweichungen für
%$\Delta N_{\mathrm{eff}}$ und $\alpha$ im Vergleich zu dem eigentlich erwarteten
%Verhalten nach dem Hamburg-Modell.
In Abbildung \ref{fig:N_eff_ohne} und \ref{fig:damage_ohne} ist das Annealingverhalten
der Dotierungskonzentration und der Schadensrate mit
diesem Ansatz dargestellt.
Das Temperaturprofil wurde während der Bestrahlung
des Sensors "R1" mit Reaktorneutronen in den
"Sandia National Laboratories" aufgenommen.
Weitere Informationen zu dem Sensor und
der Bestrahlungen finden sich in \cite{mareike}.

\begin{figure}
  \centering
    \includegraphics[width=0.82\textwidth]{build/ohnekorrektur.PDF}
    \caption{Dotierungskonzentration des Sensors R1 im Verlauf des Annealings nach einer Bestrahlung mit einer Fluenz von
    $5\cdot 10^{15} \, \mathrm{n_{eq}/cm^2}$.}
    \label{fig:N_eff_ohne}
\end{figure}

\begin{figure}
  \centering
    \includegraphics[width=0.82\textwidth]{build/damage_ohne_korrektur.PDF}
    \caption{Schadensrate des Sensors R1 im Verlauf des Annealings.}
    \label{fig:damage_ohne}
\end{figure}

%\begin{figure}
%  \centering
%  \begin{subfigure}{0.48\textwidth}
%      \centering
%      \includegraphics[height=0.82\textwidth]{build/ohnekorrektur.PDF}
%  \end{subfigure}
%  \begin{subfigure}{0.48\textwidth}
%      \centering
%      \includegraphics[height=0.82\textwidth]{build/damageohnekorrektur.PDF}
%  \end{subfigure}
%  \caption{Dotierungskonzentration (a) und Schadensrate (b) des Sensors R1 nach einer Bestrahlung mit Fluenz $\Phi_{\mathrm{eq}} = \SI{5e15}{\per\centi\meter\squared}.$}
%\end{figure}
%  \subfigure[]{\includegraphics[width=0.49\textwidth]{build/damageohnekorrektur.PDF}}
%  \caption{Dotierungskonzentration (a) und Schadensrate (b) des Sensors R1 nach einer Bestrahlung mit Fluenz $\Phi_{\mathrm{eq}} = \SI{5e15}{\per\centi\meter\squared}.$}
%  \label{fig:ohnekorrektur}
%\end{figure}

Die Dotierungskonzentration strebt nach dem Hamburger Modell
für unendlich große Zeiten gegen einen konstanten Wert und nimmt nicht mehr
ab.
Der Fehler entsteht, da Gleichung \ref{eqn:N_eff} für jeden
Zeitpunkt eine konstante Temperatur annimmt. Jedoch ist es für Annealingeffekte
relevant die Annealinghistorie zu berücksichtigen.
Das bedeutet, dass $\Delta N_{\mathrm{eff}}$ für große Zeiten abfällt, da die Gleichung über den
gesamten Zeitraum mit den zugehörigen kleinen Temperaturen rechnet. Für
die Schadensrate gilt das gleiche Problem.

Um die Effekte richtig berechnen zu können, muss eine Korrektur in den
Gleichungen vorgenommen werden.
Zunächst wird die Dotierungskonzentration betrachtet. Als Näherung wird über
jeden einzelnen Zeitabschnitt $t_{\mathrm{i}} - t_{\mathrm{i-1}}$ die
Temperatur $(T_{\mathrm{i}} +T_{\mathrm{i-1}})/2$ für das Annealing verwendet.
Da der \textit{stable damage} keine Temperaturabhängigkeit hat, muss dieser nicht
verändert werden. Für $N_{\mathrm{A}}$ wird nun folgende Näherung
verwendet:

\begin{align}
  &\frac{t_{\mathrm{n}}}{\tau_{\mathrm{a}}(T_{\mathrm{n}})} \rightarrow \sum_{i=0}^n  \frac{t_{\mathrm{i}} - t_{\mathrm{i-1}}}{\tau_{\mathrm{a}}(\frac{T_{\mathrm{i}} +T_{\mathrm{i-1}}}{2})} \:\:\:\: \text{für} \: n>0 \\
\end{align}
Für den Zeitpunkt $t=0$ gilt weiterhin $t_{\mathrm{n}}/(\tau_{\mathrm{a}}(T_{\mathrm{n}})) = 0$.
Durch die Summation der einzelnen Zeitabschnitte mit dem Mittelwert der dazugehörigen
Temperatur kann die gesamte Temperaturkurve beschrieben werden. Für $N_{\mathrm{Y}}$
wird analog die gleiche Korrektur durchgeführt.
In Abbildung \ref{fig:korrektur_N_eff} ist Dotierungskonzentration mit Korrektur dargestellt.


\begin{figure}[!htb]
  \centering
    \includegraphics[width=0.82\textwidth]{build/annealingtdata.PDF}
\caption{$\Delta N_{eff}$ des Sensors R1 mit Korrektur nach einer Bestrahlung mit einer Fluenz von $5\cdot 10^{15} \, \mathrm{n_{eq}/cm^2}$.}
\label{fig:korrektur_N_eff}
\end{figure}



Der Verlauf entspricht nun den Erwartungen, ist aber lediglich als Näherung zu verstehen, da angenommen
wird, dass die Bestrahlung vor dem Annealing stattgefunden hat. Die Sensoren wurden jedoch
während dem Annealing bestrahlt, wodurch ein genaues Berechnen der Annealingeffekte nicht
möglich ist. Für präzisere Ergebnisse sind Informationen über die zeitliche Änderung der Fluenz nötig,
zusätzlich sind die Wechselwirkungen zwischen der Fluenz und dem gleichzeitigen Annealing unklar.



%Mit der Berücksichtigung
%der einzelnen Temperaturen verzögert sich der Annealingeffekt insgesamt, da anfänglich
%geringe Temperaturen nur wenig zum gesamten Effekt beitragen. Dies wird
%an dem späteren Zeitpunkt des Minimums deutlich.

Eine solche Korrektur kann für den \textit{shortterm annealing} Term der Schadensrate ebenfalls erstellt werden.
Im zweiten Term ist $\alpha_{0}$ temperaturabhängig, aber nicht zeitabhängig,
wodurch der Korrekturansatz hier nicht mehr gelten kann. Um den Ansatz dennoch
verwenden zu können, wird die Zeit des \textit{longterm annealing} mit dem in \cite{moll} vorgeschlagenen
Skalierungsfaktor
\begin{align}
  \Theta(T) =  \exp{\left(-\frac{E_{\mathrm{I}}^*}{k_b}\left(\frac{1}{T}-\frac{1}{T_{\mathrm{ref}}}\right)\right)} \\
\end{align}
versehen, wodurch sich die Formel der Schadensrate  zu
\begin{align}
  &\alpha(t, T) = \alpha_I \cdot \exp{\left(-\frac{t}{\tau_{\mathrm{I}}(T)}\right)} + \alpha_{\mathrm{0}}^{*} -\beta \cdot \ln{\left(\Theta(T) \cdot \frac{t}{t_{\mathrm{0}}}\right)} \\
  \medskip
  \text{mit}\:\: &\alpha^*_{\mathrm{0}}(T) = \SI{-8.9e-17}{\ampere\per\centi\meter} + \SI{4.6e-14}{\ampere\kelvin\per\centi\meter} \cdot \frac{1}{T_{\mathrm{ref}}}
\end{align}
ändert.
Hierbei ist $E_{\mathrm{I}}^*$ eine Aktivierungsenergie und $\alpha^*_{0}$ nun nicht mehr temperaturabhängig,
sondern lediglich abhängig von einer Referenztemperatur $T_{\mathrm{ref}}$. Für konstante Temperaturen
ist diese Gleichung folglich äquivalent zu Gleichung \ref{eqn:damage}, für nicht
konstante Temperaturen ist die resultierende Abweichung im Vergleich zur Korrektur vernachlässigbar.
Nun kann wieder eine Korrektur analog zu den vorherigen eingeführt werden:

\begin{align}
  \Theta(T) \cdot t_{\mathrm{n}} \rightarrow \sum_{i=0}^n   \Theta \left(\frac{T_{\mathrm{i}} +T_{\mathrm{i-1}}}{2}\right) \cdot  (t_{\mathrm{i}} - t_{\mathrm{i-1}})
\end{align}

In Abbildung \ref{fig:korrektur_damage} ist die Schadensrate mit Korrektur
dargestellt.

\begin{figure}
  \centering
    \includegraphics[width=0.82\textwidth]{build/damagekorrektur.PDF}
\caption{Schadensrate des Sensors R1 mit Korrektur.}
\label{fig:korrektur_damage}
\end{figure}
%\begin{figure}
%  \begin{subfigure}[]{\linewidth}
%    \includegraphics[width=0.49\textwidth]{build/damageohnekorrektur.PDF}
%  \end{subfigure}
%  \begin{subfigure}{\linewidth}
%    \includegraphics[width=0.49\textwidth]{build/damagekorrektur.PDF}
%  \end{subfigure}[]
%  %\caption{Schadensrate des Sensors R1 ohne Korrektur (a) und mit Korrektur (b).}
%  %\label{fig:korrektur_damage}
%\end{figure}


%Auch hier zeigt das Verhalten ohne Korrektur klare Abweichungen zu den
%Erwartungen, so kann beispielsweise die Schadensrate mit voranschreitendem Annealing nicht steigen.
Mit Korrektur verhält sich die Schadensrate gemäß den Erwartungen, der stärkste
Effekt ist bei den größten Temperaturen zu sehen, kleine Temperaturen führen hingegen
nur zu minimalen Änderungen.

Mit der Implementierung dieser Korrekturen kann das Programm Annealingeffekte für
beliebige Temperaturverläuft berechnen.



\section{Lineare Interpolation der Temperatur}
Der im vorherigen Abschnitt beschriebene Korrekturansatz verwendet für jedes
Zeitintervall den Mittelwert der Anfangs- und Endtemperatur. Für große
Zeitabschnitte kommt es bei Temperaturprofilen wie bei Sandia also auch zu größeren Abweichungen von der tatsächlichen
Temperatur. Um diesen Fehler gering zu halten, wird das Temperaturprofil
linear interpoliert.
Um die Rechenzeit nicht unnötig zu erhöhen, wird eine Funktion definiert, die sinnvolle
Interpolationsschritte $n$ zwischen zwei Messpunkten abschätzt.
Da große Temperaturen relevanter für das Annealing sind und die Annealingeffekte nicht linear von
der Temperatur abhängig sind, ist es vorteilhaft,
bei den Mittelwerten der Temperaturen möglichst kleine Ungenauigkeiten und somit möglichst viele
Intervalle zu erschaffen. Dies ist insbesondere für große Temperaturänderungen von einem Messwert zum nächsten relevant.
Aus diesem Grund wird sowohl die maximale Temperatur $T_{\mathrm{max}}$ des Datensatzes
als Bezugspunkt genommen, als auch die Temperaturdifferenz zweier
aufeinander folgenden Temperaturen $T(t_{\mathrm{i}}$ und $T(t_{\mathrm{i+1}}$  betrachtet. Die Parameter $y$, $z_1$
und $z_2$ frei anpassbar sind:

\begin{align*}
  n = \left\lceil{y \cdot \frac{T(t_{\mathrm{i}})-T(t_{\mathrm{i}}+1)}{T_{\mathrm{max}}-T(t_{\mathrm{i}})+ z_1}}+z_2\right\rceil \label{eqn:intervall} .
\end{align*}
Die Funktion wächst mit steigenden Temperaturen und größer werdenden Temperaturdifferenzen.
Mit $y$ wird die Anzahl an Intervallen gesteuert, während die
Parameter $z_1$ und $z_2$ verhindern, dass $n$ null oder unendlich groß wird.
Die Funktion folgt lediglich aus den vorgegebenen Anspüchen an die Intervallzahl und beschreibt
keine optimale Abschätzung.


Für die Zeit und die Temperatur der interpolierten Daten gilt:
\begin{align}
  t_i &= t_{\mathrm{A}} +  \frac{t_{\mathrm{B}}-t_{\mathrm{A}}}{n} \cdot i \\
  T_i &= T_{\mathrm{A}} +  \frac{T_{\mathrm{B}}-T_{\mathrm{A}}}{n} \cdot i \\
  \text{mit}\:\:i &= 1, 2, ..., n
\end{align}

Hierdurch werden lineare und gleich lange Intervalle zwischen den Temperaturen
erzeugt. In den Abbildungen \ref{fig:interpolation_N_eff} und \ref{fig:interpolation_damage} sind die interpolierten Temperaturen und
die daraus berechneten Annealingeffekte dargestellt.

\begin{figure}
  \centering
    \includegraphics[width=0.8\textwidth]{build/interpolationtdata.PDF}
\caption{$\Delta N_{\mathrm{eff}}$ und zugehörige Interpolierte Temperatur für y = $15$, $z_1=\SI{5}{\kelvin}$ und $z_2=0.2$.}
\label{fig:interpolation_N_eff}
\end{figure}



\begin{figure}
  \centering
    \includegraphics[width=0.8\textwidth]{build/damage_interpolation.PDF}
\caption{$\alpha$ und zugehörige Interpolierte Temperatur für y = $15$, $z_1=\SI{5}{\kelvin}$ und $z_2=0.2$.}
\label{fig:interpolation_damage}
\end{figure}

%\begin{figure}
%  \begin{subfigure}[c]{0.5\textwidth}
%    \includegraphics[width=0.82\textwidth]{build/damage_interpolation.PDF}
%  \end{subfigure}
%  \begin{subfigure}[c]{0.5\textwidth}
%    \caption{$\Delta N_{\mathrm{eff}}$ und zugehörige Interpolierte Temperatur für y = $15$,$z_1=\SI{5}{\kelvin}$ und $z_2=0.2$.}
%  \end{subfigure}
%\caption{Zwei Bilder mit Subfigure nebeneinander}
%\end{figure}

Die Anzahl der Intervalle der linearen Interpolation steigt gemäß der Funktion \ref{eqn:intervall}
für größer werdende Temperaturen. Durch die Verwendung der Interpolation kommt es zu kleinen Abweichungen im
Vergleich zu einer Berechnung die lediglich die Messpunkte nutzt.
%Da die lineare Interpolation lediglich eine Näherung des Verlaufes der
%Temperaturkurve entspricht, kommt es bei der Aufsummierung der einzelnen Zeitabschnitte
%zu größer werdenden Abweichungen von der eigentlichen Kurve.
Die abweichenden Werte beschreiben
aufgrund der Interpolation einen präziseren Verlauf als der Annealingeffekt des ursprünglichen Datensatzes.
Jedoch beschreibt die lineare Interpolation nur eine Näherung des wahren Temperaturverlaufes.

\section{Berechnung der gesamten Annealinghistorie eines Sensors}
Die hier betrachtete Annealinghistorie setzt sich aus 60 Erwärmungs-und
Abkühlungszyklen des Sensors "P3" zusammen, welche insgesamt $\SI{1800}{\minute}$ entsprechen.
Für den Sensor "P4" werden 39 Zyklen ($\SI{1170}{\minute}$) betrachtet.
Beide wurden von der ${\mathrm{IRRAD}}$ Einrichtung mit $\SI{24}{\giga\eV\per\clight}$ Protonen des Protonen Synchrotrons bestrahlt.
Die Datensätze werden mithilfe eines Python Programmes miteinander Verbunden,
wobei die Zeiten zwischen dem Annealing von einem Zyklus zum nächsten auf null gesetzt werden. Wegen der geringen
Temperaturen nach jeder Abkühlung sind die daraus resultierenden Abweichungen vernachlässigbar.
Für jeden Zyklus wurde bei den Sensoren eine Temperatur von $\SI{60}{\celsius}$ angestrebt.
Mehr Informationen zu den Eigenschaften von P3 und P4 in  \cite{felix}.


In Abbildung \ref{fig:P_3} ist die Schadensrate für die 60 Annealingzyklen,
sowie die experimentell bestimmte Schadensrate dargestellt.

\begin{figure}
  \centering
    \includegraphics[width=0.82\textwidth]{build/damage_ohne_temperatur.PDF}
\caption{Theoretische Schadensrate und gemessene Schadensrate nach 1, 2, 3, 4, 6, 9, 14, 24, 39 und 60 Zyklen .}
\label{fig:P_3}
\end{figure}

Der stufenartige Verlauf der Kurve resultiert aus den Temperaturzyklen.
Es ist eine merkliche Abweichung der experimentellen Messwerte von den theoretischen
Werten erkennbar. Der Sensor "P3" wurde schon vor der Messung unkontrolliert erwärmt, wodurch
die angegebenen Annealingzeiten in Wahrheit größer sind und nicht präzise abgeschätzt werden können. Dies erklärt die
geringe Änderung der einzelnen Messwerte voneinander, welche dem
Verlauf der theoretischen Werte für große Zeiten ähnelt.
Unter Berücksichtigung dieser Erwärmung verschieben sich die Messpunkte im Vergleich zu den
simulierten Werten zu höheren Annealingzeiten.
%Die Abweichung der theoretischen
%und experimentellen Funktionswerte wird dadurch größer.
Für die Bestimmung der Schadensrate
wird das Detektorvolumen benötigt, außerdem nimmt das Modell den Leckstrom bei $\SI{20}{\celsius}$ an.
%Bei der Messung wurde
%$\Delta I$ mit dem Leckstrom gleichgesetzt. Dies gilt nur bei einer Temperatur
%von $\SI{20}{\celsius}$,
Bei der Messung des Stroms betrug die tatsächliche Temperatur
$\SI{-2}{\celsius}$. Um dies zu korrigieren wird der Leckstrom gemäß \ref{eqn:Chilingarov_2013}
mit der Temperatur skaliert. Die Aktivierungsenergie $E_{\mathrm{eff}}$ und das Volumen des Detektormaterials
unterliegen dabei Unsicherheiten. Zusätzlich haben auch die materialspezifischen Parameter
der theoretischen Messwerte eine gewisse Abweichung zu den tatsächlichen Parametern, da
ein anderer Sensor betrachtet wird.

Die Messung des Sensors "P4" ist in Abbildung \ref{fig:P_4}
dargestellt.

\begin{figure}
  \centering
    \includegraphics[width=0.82\textwidth]{build/damage_P_4.PDF}
\caption{Theoretische Schadensrate und gemessene Schadensrate nach 1, 2, 3, 4, 6, 9, 14, 24 und 39 Zyklen.}
\label{fig:P_4}
\end{figure}

Die Abweichungen sind in dieser Messreihe noch deutlicher. Neben den Fehlerquellen die
für den Sensor "P3" aufgezählt wurden, ist "P4" zusätzlich nicht vollständig depletiert.
Für die Bestimmung des aktiven Volumens zur Berechnung der Schadensrate wurde ein vollständig
depletierter Sensor angenommen. Das tatsächliche aktive Volumen des Detektors ist somit kleiner, wodurch sich auch
die gemessene Schadensrate verringert.
Um diesen Effekt berücksichtigen zu können, müsste das Depletierte Volumen
beispielweise über Ladungssammlungseffizienz (CCE) Messungen bestimmt werden.
%Die Funktion \ref{eqn:damage} kann solche Effekte nicht berücksichtigen.
