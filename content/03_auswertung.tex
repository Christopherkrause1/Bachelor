\chapter{Auswertung}\label{make}
\section{Annealingeffekt für konstanten Temperaturen}
Gleichung \ref{eqn:N_eff} und \ref{eqn:damage}  wird für das Modellieren der Annealingeffekte verwendet. In
Abbildung \ref{fig:N_eff} und \ref{fig:damage} sind diese, jeweils für zwei Temperaturen dargestellt.
Für alle Plots werden materialabhängige
Parameter einer "WE-25k" aus \cite{moll} genommen.

\begin{figure}
    \includegraphics[width=0.95\textwidth]{build/annealing.PDF}
    \caption{$\Delta N_{\mathrm{eff}}$ einer WE-25k Diode nach Bestrahlung mit Fluenz $\Phi_{\mathrm{eq}} = \SI{5e15}{\per\centi\meter\squared}.$}
    \label{fig:N_eff}
\end{figure}

\begin{figure}
    \includegraphics[width=0.95\textwidth]{build/damage.PDF}
    \caption{Schadensrate einer WE-25k Diode.}
    \label{fig:damage}
\end{figure}

[...]



\section{Annealingeffekt für nicht konstanten Temperaturen}
Werden die selben Gleichungen zur Modellierung von Annealingeffekte für nicht
konstante Temperaturen verwendet, so kommt es zu deutlichen Abweichungen für
$\Delta N_{\mathrm{eff}}$ und $\alpha$ im Vergleich zu dem eigentlich erwarteten
Verhalten nach dem Hamburg-Modell. In Abbildung \ref{fig:ohnekorrektur} ist das Annealingverhalten
der Dotierungskonzetration (a) und der Schadensrate (b) mit
diesem Ansatz dargestellt. Das Temperaturprofil wurde von einem Sensor "R1" in
"Sandia National Laboratories" aufgenommen.

%\begin{figure}
%  \subfigure[]{\includegraphics[width=0.49\textwidth]{build/ohnekorrektur.PDF}}
%  \subfigure[]{\includegraphics[width=0.49\textwidth]{build/damageohnekorrektur.PDF}}
%  \caption{Dotierungskonzentration (a) und Schadensrate (b) des Sensors R1 nach einer Bestrahlung mit Fluenz $\Phi_{\mathrm{eq}} = \SI{5e15}{\per\centi\meter\squared}.$}
%  \label{fig:ohnekorrektur}
%\end{figure}

Der longterm Annealingeffekt der Dotierungskonzetration soll nach dem Hamburg-Modell
für unendlich große Zeiten gegen einen konstanten Wert streben und nicht mehr
kleiner werden können. Die Schadensrate kann mit fortschreitendem annealing nicht
größer werden, somit Verhalten sich beide Kurven nicht dem theoretisch
vorhergesehenem Verhalten.
Der Fehler entsteht, da die Gleichung \ref{eqn:N_eff} und \ref{eqn:damage} für jeden
Zeitpunkt eine konstante Temperatur annehmen. Jedoch ist es für Annealingeffekte
wichtig mit welchen Temperaturen und wie lange damit vorher annealed wurde.
Das bedeutet, dass $\Delta N_{\mathrm{eff}}$ für große Zeiten abfällt, da die Gleichung über den
gesamten Zeitraum mit den zugehörigen kleinen Temperaturen rechnet. Für
die Schadensrate gilt das gleiche Problem.

Um die Effekte richtig berechnen zu können, muss eine Korrektur in den
Gleichungen vorgenommen werden.
Zunächst wird die Dotierungskonzentration betrachtet. Als Näherung soll über
jeden einzelnen Zeitabschnitt $t_{\mathrm{i}} - t_{\mathrm{i-1}}$ mit der
Temperatur $\frac{T_{\mathrm{i}} +T_{\mathrm{i-1}}}{2}$ annealed werden.
Da der stable damage keine Temperaturabhängigkeit hat muss dieser nicht
verändert werden. Für $N_{\mathrm{A}}$ soll nun folgende Näherung
verwendet werden:

\begin{align}
  &\frac{t_{\mathrm{n}}}{\tau_{\mathrm{a}}(T_{\mathrm{n}})} \rightarrow \sum_{i=0}^n  \frac{t_{\mathrm{i}} - t_{\mathrm{i-1}}}{\tau_{\mathrm{a}}(\frac{T_{\mathrm{i}} +T_{\mathrm{i-1}}}{2})} \:\:\:\: \text{für} \: n>0 \\
\end{align}
Für den Zeitpunkt $t=0$ gilt weiterhin $\frac{t_{\mathrm{n}}}{\tau_{\mathrm{a}}(T_{\mathrm{n}})} = 0$.
Durch die Summation der einzelnen Zeitabschnitte mit dem Mittelwert der dazugehörigen
Temperatur kann die gesamte Temperaturkurve beschrieben werden. Für $N_{\mathrm{Y}}$
wird analog die gleiche Korrektur durchgeführt.
In Abbildung \ref{fig:korrektur_N_eff} ist Dotierungskonzentration ohne und mit Korrektur dargestellt.

%\begin{figure}
%    \subfigure[]{\includegraphics[width=0.49\textwidth]{images/ohnekorrektur.PDF}}
%    \subfigure[]{\includegraphics[width=0.49\textwidth]{images/annealingtdata.PDF}}
%    \caption{$\Delta N_{eff}$ des Sensors R1 ohne Korrektur (a) und mit Korrektur (b) nach einer Bestrahlung mit Fluenz $\Phi_{\mathrm{eq}} = \SI{5e15}{\per\centi\meter\squared}.$}
%    \label{fig:korrektur_N_eff}
%\end{figure}


[...]

Eine solche Korrektur kann für den shortterm annealing Term der Schadensrate ebenfalls gemacht werden.
Für den zweiten Term ist $\alpha_{0}$ temperaturabhängig, aber nicht zeitabhängig,
wodurch der Korrekturansatz hier nicht mehr gelten kann. Um den Ansatz dennoch
verwenden zu können, wird die Zeit des longterm annealing [Grammatik] mit dem
Skalierungsfaktor
\begin{align}
  $\Theta(T) = \frac{E_{\mathrm{I}}^*}{k_b} \exp{\left(-\frac{1}{T}-\frac{1}{T_{\mathrm{ref}}}\right)}$
\end{align}
versehen. Dadurch ändert sich auch $\alpha_{0}$ und ist nun nicht mehr temperaturabhängig,
sondern lediglich abhängig von einer Referenztempeeratur $T_{\mathrm{ref}}$. Für konstante Temperaturen
ist diese Gleichung folglich äquivalent zu Gleichung \ref{eqn:damage}, für nicht
konstante Temperaturen ist sie eine Näherung.
