\chapter{Ergebnisse und Ausblick}
Das Programm beschreibt die Dotierungskonzentration und die Schadensrate
auf der Grundlage des Hamburger Modells präzise. Die Näherung für beliebige
Temperaturen verursacht für hinreichend viele Interpolationsintervalle
vernachlässigbar kleine Fehler. Das Modell versagt für kleine Schadensraten $\alpha < \SI{1.5e-17}{\ampere\per\centi\meter}$,
da diese nicht beliebig gering werden können. Solche Werte werden jedoch erst nach
einer Annealingzeit von mehr als $t=\SI{1e4}{\minute}$ bei $\SI{80}{\celsius}$ erreicht, wofür
die Dotierungskonzentration bereits nahe am Sättigungswert liegt.

Die Verbindung von Datensätzen und die Anknüpfung der einzelnen Zeitstempel der
Textdateien funktioniert unter vernachlässigbar kleinen Ungenauigkeiten, da das annealing
nach einem Abkühlungszyklus selbst über große Zeiträume irrelevant im Vergleich zu
den Erwärmungszyklen.

Abweichungen von experimentellen Daten sind durch systematische Fehler in der Messung
zu erklären. Unvollständige Depletion des Sensors, vorheriges annealing über einen
unbekannten Zeitraum, und Unsicherheiten in der Temperaturskalierung des Leckstroms sowie
des Volumens des Detektors sind dabei die Hauptursachen.

Das Interface modelliert Annealingeffekte für beliebige Fluenzen, Temperaturen und
Zeiten zuverlässig. Die dargestellten Werte entsprechen den genauen Funktionswerten
der Funktionen \ref{eqn:N_eff} und \ref{eqn:damage}, da keine Näherungen nötig sind.


Das erstellte Programm kann nur Annealingeffekte mit konstanter Fluenz
bestimmen. Die präzise Modellierung des Annealingverhaltens während der Bestrahlung
ist hiermit nicht möglich. Anders als die Temperaturen, die mit einem Zeitabschnitt
assoziiert werden, treten die Fluenzen in der Dotierungskonzentration
unabhängig von der Zeit auf, weshalb eine andere Herangehensweise nötig ist.
