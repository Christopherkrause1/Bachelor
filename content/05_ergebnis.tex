\chapter{Ergebnisse und Ausblick}
Das Programm beschreibt die Dotierungskonzentration und die Schadensrate während dem
Annealing einer bestrahlten Diode
auf der Grundlage des Hamburger Modells. Die Näherung für beliebige
Temperaturen verursacht für hinreichend viele Interpolationsintervalle
vernachlässigbar kleine Fehler. Die Anwendbarkeit des Modells zur Berechnung der Schadensrate ist
beschränkt, so würden große Annealingzeiten und Temperaturen ($\alpha < \SI{1.5e-17}{\ampere\per\centi\meter}$) zu Abweichungen
zwischen Modell und Realität führen. Solche Werte werden jedoch erst nach
einer Annealingzeit von mehr als $t=\SI{1e4}{\minute}$ bei $\SI{80}{\celsius}$ erreicht, wofür
die Dotierungskonzentration bereits nahe am Sättigungswert liegt.
%versagt für kleine Schadensraten $\alpha < \SI{1.5e-17}{\ampere\per\centi\meter}$,
%da diese nicht beliebig gering werden können. Solche Werte werden jedoch erst nach
%einer Annealingzeit von mehr als $t=\SI{1e4}{\minute}$ bei $\SI{80}{\celsius}$ erreicht, wofür
%die Dotierungskonzentration bereits nahe am Sättigungswert liegt.

Die Verbindung von Datensätzen und die Anknüpfung der einzelnen Zeitstempel der
Textdateien funktioniert unter vernachlässigbar kleinen Ungenauigkeiten, da das Annealing
nach einem Abkühlungszyklus selbst über große Zeiträume irrelevant im Vergleich zu
den Erwärmungszyklen ist.

Abweichungen von experimentellen Daten sind durch systematische Fehler in der Messung
zu erklären. Unvollständige Depletion des Sensors, vorheriges Annealing über einen
unbekannten Zeitraum, und Unsicherheiten in der Temperaturskalierung des Leckstroms sowie
des Volumens des Detektors sind dabei die Hauptursachen.

Es wurde ein Interface erstellt, dass das Modell nutzt um die Schadensrate und die Dotierungskonzentration
für beliebige Fluenzen, Temperaturen und
Zeiten vorherzusagen. Die dargestellten Werte entsprechen den genauen Funktionswerten
der Funktionen \ref{eqn:N_eff} und \ref{eqn:damage}, da keine Näherungen nötig sind.


Das erstellte Programm kann nur Annealingeffekte mit konstanter Fluenz
bestimmen. Die präzise Modellierung des Annealingverhaltens während der Bestrahlung
ist hiermit nicht möglich.
Aus diesem Grund ist eine Aussage über die Genauigkeit der Modellierung der
Annealingeffekte für das Temperaturprofil aus Sandia schwierig, da dieses während der Bestrahlung
aufgenommen wurde.
Um nicht konstante Fluenzen betrachten zu können, müssen Annahmen
über die Änderung der Fluenz getroffen werden, welche für das
hier betrachtete Temperaturprofil nicht bekannt sind.

%Anders als die Temperaturen, die mit einem Zeitabschnitt
%assoziiert werden, treten die Fluenzen in der Dotierungskonzentration
%unabhängig von der Zeit auf, weshalb eine andere Herangehensweise nötig ist.
