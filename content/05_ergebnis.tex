\chapter{Zusammenfassung}
Im Zuge dieser Arbeit wurde ein Programm entwickelt, dass
die Dotierungskonzentration und die Schadensrate während dem
Annealing einer bestrahlten Diode
auf der Grundlage des Hamburger Modells beschreibt.

Durch beliebig viele Interpolationsintervalle im Temperaturprofil kann die entstehende Abweichung des
Korrekturansatzes klein gehalten werden.
%Die Näherung für beliebige
%Temperaturverläufe verursacht für hinreichend viele Interpolationsintervalle
%vernachlässigbar kleine Fehler.

Für große Annealingzeiten beschreibt Gleichung \ref{eqn:damage} nicht mehr die tatsächliche Schadensrate, wodurch das Programm,
welches auf dieser Gleichung basiert, dies ebenfalls nicht kann. Abweichungen
von dem Hamburger Modell werden jedoch erst nach
einer Annealingzeit von mehr als $t=\SI{e4}{\minute}$ bei $\SI{80}{\celsius}$ erreicht (siehe
\ref{fig:damage_rates}). Für solche Werte liegt
die Dotierungskonzentration bereits nahe am Sättigungswert.

%Die Anwendbarkeit des Modells zur Berechnung der Schadensrate ist
%beschränkt, so würden große Annealingzeiten und Temperaturen ($\alpha < \SI{1.5e-17}{\ampere\per\centi\meter}$) zu Abweichungen
%zwischen Modell und Realität führen. Solche Werte werden jedoch erst nach
%einer Annealingzeit von mehr als $t=\SI{e4}{\minute}$ bei $\SI{80}{\celsius}$ erreicht (siehe
%\ref{fig:damage_rates}), wofür
%die Dotierungskonzentration bereits nahe am Sättigungswert liegt.
%versagt für kleine Schadensraten $\alpha < \SI{1.5e-17}{\ampere\per\centi\meter}$,
%da diese nicht beliebig gering werden können. Solche Werte werden jedoch erst nach
%einer Annealingzeit von mehr als $t=\SI{1e4}{\minute}$ bei $\SI{80}{\celsius}$ erreicht, wofür
%die Dotierungskonzentration bereits nahe am Sättigungswert liegt.

%Die Verbindung von Datensätzen und die Anknüpfung der einzelnen Zeitstempel der
%Textdateien funktioniert.
%Abweichungen von experimentellen Daten führen zu vernachlässigbar kleinen Ungenauigkeiten, da das Annealing
%nach einem Abkühlungszyklus selbst über große Zeiträume irrelevant im Vergleich zu
%den Erwärmungszyklen ist.

Abweichungen von experimentellen Daten sind durch systematische Unsicherheiten der Messdaten
zu erklären. Unvollständige Depletion des Sensors, vorheriges Annealing über einen
unbekannten Zeitraum und Unsicherheiten in der Temperaturskalierung des Leckstroms sind dabei die Hauptursachen.

Es wurde ein Interface erstellt, dass das Modell nutzt um die Schadensrate und die Dotierungskonzentration
für beliebige Fluenzen, Temperaturen und
Zeiten vorherzusagen. Die dargestellten Werte entsprechen den genauen Funktionswerten
der Funktionen \ref{eqn:N_eff} und \ref{eqn:damage}, da keine Näherungen nötig sind.

Das erstellte Programm kann nur Annealingeffekte mit konstanter Fluenz bestimmen.
Die präzise Modellierung des Annealingverhaltens während der Bestrahlung ist
hiermit nicht möglich. Aus diesem Grund ist eine Aussage über die Genauigkeit der
Modellierung der Annealingeffekte für das Temperaturprofil des in Sandia bestrahlten Sensors nicht möglich.

%dieses während der Bestrahlung aufgenommen wurde. Um nicht konstante Fluenzen
%betrachten zu können, müssen Annahmen über den zeitlichen Verlauf der Bestrahlung getroffen
%werden, welche für das hier betrachtete Temperaturprofil nicht bekannt sind.
%Zudem ist die Wechselwirkung der Bestrahlung und dem zeitgleichen Annealing unklar.
%Das erstellte Programm kann nur Annealingeffekte mit konstanter Fluenz
%bestimmen. Die präzise Modellierung des Annealingverhaltens während der Bestrahlung
%ist hiermit nicht möglich.
%Aus diesem Grund ist eine Aussage über die Genauigkeit der Modellierung der
%Annealingeffekte für das Temperaturprofil aus Sandia nicht möglich.


%Anders als die Temperaturen, die mit einem Zeitabschnitt
%assoziiert werden, treten die Fluenzen in der Dotierungskonzentration
%unabhängig von der Zeit auf, weshalb eine andere Herangehensweise nötig ist.
