\chapter{Interface zur Berechnung von Annealingeffekten für konstante Temperaturen}
Das Hauptprogramm dient zur Modellierung von Annealingeffekten für beliebige Datensätze.
Um den erwarteten Verlauf von Annealingeffekten von konstanten Temperaturen über einen
beliebigen Zeitraum bestimmen zu können dient ein weiteres Python Programm diesem Zweck.
Bei der Ausführung öffnet es ein Interface, welches nach den relevanten Angaben für
das Modellieren der Dotierungskonzentration oder der Schadensrate fragt. In Abbildung \ref{fig:interface}
ist das Interface dargestellt.

\begin{figure}
    \includegraphics[width=0.83\textwidth]{build/interface.PNG}
\caption{Interface zur Berechnung von $N_{\mathrm{eff}}$ und $\alpha$.}
\label{fig:interface}
\end{figure}

Die Berechnung erfolgt dabei ebenfalls auf der Grundlage des Hamburger Modells und modelliert die
Annealingeffekte von
$0,1$ Sekunden bis zu dem angegebenen Zeitintervall.
Das Zeitintervall von einem Wert zum nächsten ist dabei logarithmisch, wodurch über alle
Größenordnungen der Zeit hinweg der Abstand auf der logarithmischen Zeitachse gleich bleibt.
Die Parameter können in den Einstellungen beliebig gewählt werden.

In Abbildung \ref{fig:inter_n} und \ref{fig:inter_damage} sind Beispielhaft jeweils ein Plot für die Dotierungskonzentration und der
Schadensrate dargestellt. Hierbei werden die Parameter einer "WE-25k" Diode verwendet.


\begin{figure}
  \includegraphics[width=0.83\textwidth]{build/interface_n_eff.PDF}
  \caption{Beispiel Plot von $N_{\mathrm{eff}}$ mit $t = \SI{e4}{\minute}$, $\Phi_{\mathrm{eq}}= 5\cdot 10^{15} \, \mathrm{n_{eq}/cm^2}$ und $T=\SI{60}{\celsius}$.}
  \label{fig:inter_n}
\end{figure}

\begin{figure}
  \includegraphics[width=0.83\textwidth]{build/interface_damage.PDF}
  \caption{Beispiel Plot von $\alpha$ mit $t = \SI{e4}{\minute}$ und $T=\SI{60}{\celsius}$.}
  \label{fig:inter_damage}
\end{figure}
